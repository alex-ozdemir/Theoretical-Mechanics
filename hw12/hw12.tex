\documentclass[11pt,letterpaper,boxed]{../hmcpsetrhino}
\usepackage[margin=1in]{geometry}
\usepackage{graphicx}
\usepackage{enumerate}
\usepackage{amsthm}
\usepackage{amsmath}

\newcommand{\ds}{\displaystyle}
\newcommand{\half}{\frac{1}{2}}
\newcommand*\Eval[3]{\left.#1\right\rvert_{#2}^{#3}}
\newcommand{\eval}{\biggr\rvert}
\newcommand\Partial[2]{\frac{\partial #1}{\partial #2}}
\newcommand\mat[1]{\underline{\vec {#1}}}
\newcommand\tp[1]{\widetilde {#1}}
\def\EE{{\cal E}}
\def\Lagr{\mathcal{L}}
\def\Ham{\mathcal{H}}

\name{}
\class{Physics 111 Section 1}
\assignment{Problem Set 12}
\duedate{October 26, 2016}

\begin{document}

\problemlist{Rotational Motion: Principal Axes}
\textbf{Help:} \\
\begin{problem}[i]
Kinetic energy and the inertia tensor. This problem is a little heavy on vector algebra for my taste, but the result is important and we will rely on it later. So...

\begin{problem}[10.33]
Here is a good exercise in vector identities and matrices, leading to some important general results:
\begin{enumerate}[(a)]
\item For a rigid body made up of particles of mass $m_\alpha$, spinning about an axis through the origin with angular velocity $\omega$, prove that its total kinetic energy can be written as 
\[	T = \half \sum m_\alpha \left[ (\omega r_\alpha)^2 - (\vec \omega \cdot \vec r_\alpha)^2 \right]\]
Remember that $\vec v_\alpha = \vec \omega \times \vec r_\alpha$. You may find the following vector identity useful: For any two vectors $\vec a$ and $\vec b$,
\[	(\vec a \times \vec b)^2 = a^2 b^2 - (\vec a \cdot \vec b)^2 \]
(If you use the identity, please prove it.)

\item Prove that the angular momentum $\vec L$ of the body can be written as 
\[	\vec L = \sum m_\alpha \left[\vec \omega r_\alpha^2 - \vec r_\alpha (\vec \omega \cdot \vec r_\alpha)\right]\]

For this you will need the so-called $B A C - CAB$ rule, that $\vec A \times (\vec B \times \vec C) = \vec B(\vec A \cdot \vec C) - \vec C(\vec A \cdot B)$.

\item Combine the results of parts (a) and (b) to prove that 
\[	T = \half \vec \omega \cdot \vec L = \half \widetilde{\vec \omega} \vec I \vec \omega\]
Prove both equalities. The last expression is a matrix product: $\vec \omega$ denotes the $3 \times 1$ column of numbers $\omega_x, \omega_y, \omega_z$, the tilde on $\widetilde{ \vec \omega}$ denotes the matrix transpose (in this case a row), and $\vec I$ is the moment of inertia tensor. This result is actually quite important; it corresponds to the much more obvious result that for a particle, $T = \half \vec v \cdot \vec p$. 

\item Show that with respect to the principal axes, $T = \half (\lambda_1 {\omega_1}^2 + \lambda_2 {\omega_2}^2 + \lambda_3 {\omega_3}^2)$, as in Equation (10.68).
\end{enumerate}

\end{problem}
\end{problem}
\begin{solution}

\vfill
\end{solution}

\newpage 

\begin{problem}[ii]
Properties of eigenvectors and eigenvalues for symmetric, real matrices. This is a problem that you probably already did in linear algebra (feel free to revisit your notes, problem sets, and text from that class). However, it's worth reminding yourself of these results; we will make use of them going forward, and you will use them next semester in Big Quantum (116) all of the time.

\begin{problem}[10.38]
Suppose that you have found three independent principal axes (directions $\vec e_1, \vec e_2, \vec e_3$) and corresponding principal moments $\lambda_1, \lambda_2, \lambda_3$ of a rigid body whose moment of inertia tensor $\vec I$ (not diagonal) you had calculated. (You may assume, what is actually fairly easy to prove, that all of the quantities concerned are real.)
\begin{enumerate}[(a)]
\item Prove tha if $\lambda_i \neq \lambda_j$ then it is automatically the case that $\vec e_i \cdot \vec e_j = 0$. (It may help to introduce a notation that distinguishes between vectors and matrices. For example, you could use an underline to indicate a matrix, so that $\mat a$ is the $3 \times 1$ matrix that represents the vector $\vec a$, and the vector scalar product $\vec a \cdot \vec b$ is the same as the matrix product $\tp{\mat a}\,  \mat b$ or $\tp{\mat b}\, \mat a$. Then consider the number $\tp{\mat e}_i\, \mat I \, \mat e_j$, which can be evaluated in two ways using the fact that both $\vec e_i$ and $\vec e_j$ are eigenvectors of $\vec I$.)

\item Use the result of part (a) to show that if the three principal moments are all different, then the directions of three principal axes are uniquely determined. 

\item Prove that if two of the principal moments are equal, $\lambda_1 = \lambda_2$ say, then any direction in the plane of $\vec e_1$ and $\vec e_2$ is also a principal axis with the same principal moment. In other words, when $\lambda_1 = \lambda_2$ the corresponding principal axes are not uniquely determined.

\item Prove that if all three principal moments are equal, then \textit{any} axis is a principal axis with the same principal moment.
\end{enumerate}
\end{problem}
\end{problem}
\begin{solution}

\vfill
\end{solution}


\end{document}
