\documentclass[11pt,letterpaper,boxed]{../hmcpsetrhino}
\usepackage[margin=1in]{geometry}
\usepackage{graphicx}
\usepackage{enumerate}
\usepackage{amsthm}
\usepackage{amsmath}

\newcommand{\ds}{\displaystyle}
\newcommand{\half}{\frac{1}{2}}
\newcommand*\Eval[3]{\left.#1\right\rvert_{#2}^{#3}}
\newcommand{\eval}{\biggr\rvert}
\newcommand\Partial[2]{\frac{\partial #1}{\partial #2}}
\let\oldvec\vec
\renewcommand{\vec}[1]{\oldvec{\mathbf{#1}}}
\def\EE{{\cal E}}
\def\Lagr{\mathcal{L}}
\def\Ham{\mathcal{H}}

\name{}
\class{Physics 111 Section 1}
\assignment{Problem Set 10}
\duedate{October 12, 2016}

\begin{document}

\problemlist{Noninertial Frames, Examples}
\textbf{Help:} 

\begin{problem}[i]
Visualize the movement of a mass on a turntable.

\begin{problem}[9.20]
Consider a frictionless puck on a horizontal turntable that is rotating counterclockwise with angular velocity $\Omega$. 
\begin{enumerate}[(a)]
\item Write down Newton's second law for the coordinates $x$ and $y$ of the puck as seen by me standing on the turntable. (Be sure to include the centrifugal and Coriolis forces, but ignore the earth's rotation.)

\item Solve the two equations by the trick of writing $\eta = x + i y$ and quessing a solution of the form $\eta = e^{-i\alpha t}$. [In this case - as in the case of critically damped SHM discussed in Section 5.4 - you get only one solution this way. The other has the same form (5.43) we found for the second solution in damped SHM.] Write down the general solution. 

\item At time $t = 0$, I push the puck from position $\vec r_0 = (x_0,0)$ with velocity $ \vec v_0 = (v_{x0}, v_{y0})$ (all as measured by me on the turntable). Show that 
\begin{align*}
x(t) &= (x_0 + v_{x0}t) \cos \Omega t + (v_{y0} + \Omega x_0)t \sin\Omega t\\
y(t)&= -(x_0 + v_{x0}t) \sin \Omega t + (v_{y0} + \Omega x_0) t \cos \Omega t
\end{align*}

\item Describe and sketch the behavior of the puck for large values of $t$.[\textit{Hint}: when $t$ is large the terms proportional to $t$ dominate (except in the case that both their coefficients are zero). With $t$ large, write (9.72) in the form $x(t) = t(B_1 \cos \Omega t+ B_2 \sin \Omega t)$, with a similar expression for $y(t)$, and use the trick of (5.11) to combine the sine and cosine into a single cosine - or sine, in the case of $y(t)$. By now you can recognize that the path is the same kind of spiral, whatever the initial conditions (with the one exception mentioned).]
\end{enumerate}
\end{problem}
\end{problem}
\begin{solution}


\vfill
\end{solution}

\newpage 

\begin{problem}[ii]
The technique of successive approximations is useful.

\begin{problem}[9.26]
In Section 9.8, we used a method of successive approximations to find the orbit of an object that is dropped from rest, correct to first order in the earth's angular velocity $\Omega$. Show in the same way that if an object is thrown with initial velocity $\vec v_0$ from a point $O$ on the earth's surface at colatitude $\theta$, then to first order in $\Omega$ its orbit is
\begin{align*}
x &= v_{x0} t + \Omega (v_{y0} \cos \theta - v_{z0} \sin \theta )t^2 + \frac{1}{3}\Omega g t^3 \sin \theta \\
y &= v_{y0}t - \Omega (v_{x0} \cos \theta) t^2\\
z &= v_{z0} t - \half g t^2 + \Omega (v_{x0} \sin \theta ) t^2
\end{align*}
[First solve the equations of motion (9.53) in zeroth order, that is, ignoring $\Omega$ entirely. Substitute your zeroth-order solution for $\dot x, \dot y$, and $\dot z$ into the right side of equations (9.53) and integrate to give the next approximation. Assume that $v_0$ is small enough that air resistance is negligible and that $\vec g$ is a constant throughout the flight.]

\end{problem}
\end{problem}
\begin{solution}


\vfill
\end{solution}


\newpage

\begin{problem}[iii]
Naval history and the Coriolis force.

\begin{problem}[9.28]
Use the result (9.73) of Problem 9.26 to do the following: A naval gun shoots a shell at colatitude $\theta$ in a direction that is $\alpha$ above the horizontal and due east, with muzzle speed $v_0$. 
\begin{enumerate}
\item Ignoring the earth's rotation (and air resistance), find how long (t) the shell would be in the air and how far away (R) it would land. If $v_0 = 500$m/s and $\alpha = 20^{\circ}$, what are $t$ and $R$? 

\item A naval gunner spots and enemy ship due east at the range $R$ of part (a) and , forgetting about the Coriolis effect, aim his gun exactly as in part (a). Find by how far north or south, and in which direction, the shell will miss the target, in terms of $\Omega$, $v_0$, $\alpha$, $\theta,$ and $g$. (It will also miss in the east-west direction but this is perhaps less critical.) If the incident occurs at latitude 50$^\circ$ north ($\theta = 40^\circ$), what is the distance? What if the latitude is 50$^\circ$ south? This problem is a serious issue in long-range gunnery: In a battle near the Falkland Islands in World War I, the British navy consistently missed German ships by many tens of yards because they apparently forgot that the Coriolis effect in the southern hemisphere is opposite to that in the north.
\end{enumerate}
\end{problem}
\end{problem}
\begin{solution}


\vfill
\end{solution}

\end{document}
