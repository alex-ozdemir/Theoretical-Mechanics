\documentclass[11pt,letterpaper,boxed]{../hmcpsetrhino}
\usepackage[margin=1in]{geometry}
\usepackage{graphicx}
\usepackage{enumerate}
\usepackage{amsthm}
\usepackage{amsmath}

\newcommand{\ds}{\displaystyle}
\newcommand{\half}{\frac{1}{2}}
\newcommand*\Eval[3]{\left.#1\right\rvert_{#2}^{#3}}
\newcommand{\eval}{\biggr\rvert}
\newcommand\Partial[2]{\frac{\partial #1}{\partial #2}}
\let\oldvec\vec
\renewcommand{\vec}[1]{\oldvec{\mathbf{#1}}}
\def\EE{{\cal E}}
\def\Lagr{\mathcal{L}}
\def\Ham{\mathcal{H}}

\name{}
\class{Physics 111 Section 1}
\assignment{Problem Set 06}
\duedate{September 28, 2016}

\begin{document}

\problemlist{Energy \& Conservative Forces}
\textbf{Help:} 

\begin{problem}[i]
Gravity on Planet X.\\
\begin{problem}[4.7]
Near to the point where I am standing on the surface of Planet X, the gravitational force on a mass $m$ is vertically down but has magnitude $m \gamma y^2$ where $\gamma$ is a constant and $y$ is the mass's height above the horizontal ground. 
\begin{enumerate}[(a)]
\item Find the work done by gravity on a mass $m$ moving from $\vec r_1$ to $\vec r_2$, and use your answer to show that gravity on Planet X, although most unusual, is still conservative. find the corresponding potential energy.
\item Still on the same planet, I thread a bead on a curved, frictionless, rigid wire, which extends from ground level to a height $h$ above the ground. Show clearly in a picture the forces on the bead when it is somewhere on the wire. (Just name the forces so it's clear what they are; don't worry about their magnitude.) Which of the forces are conservative and which are not?
\item If I release the bead from rest at a height $h$, how fast will it be going when it reaches the ground?
\end{enumerate}
\end{problem}
\vspace{-0.45cm}
\end{problem}

\begin{solution}

\vfill
\end{solution}

\newpage 

\begin{problem}[ii]
Assume that $\nabla \times \vec F = 0$. With your knowledge of calculus, including Stokes' theorem, show that this statement is equivalent to writing the following:

\begin{enumerate}[(a)]
\item $\vec F = - \vec{\nabla} U$

\item $\int_1^2 \vec{F} \cdot \vec{ds} = \text{path independent}$

\item $\oint \vec F \cdot \vec{ds} = 0$

\end{enumerate}

\end{problem}

\begin{solution}

\vfill
\end{solution}


\newpage

\begin{problem}[iii]
Time-dependent forces and their relationship to conservative forces.
\hfill\\
\begin{problem}[4.27]
Suppose that the force $\vec F(\vec r, t)$ depends on the time $t$ but still satisfies $\nabla \times \vec F = 0$. It is a mathematical fact (related to Stokes' theorem as discussed in Problem 4.25) that the work integral $\int_1^2 \vec F(\vec r, t) \cdot d\vec r$ (evaluated at any one time $t$) is independent of the path taken between the points 1 and 2. Use this to show that the time-dependent PE defined by (4.48), for any fixed time $t$, has the claimed property that $\vec F(\vec r, t) = - \nabla U(\vec r, t)$. Can you see what goes wrong with the argument leading to Equation (4.19), that is, conservation of energy?
\end{problem}
\vspace{-0.45cm}
\end{problem}

\begin{solution}

\vfill
\end{solution}


\end{document}
