\documentclass[11pt,letterpaper,boxed]{../hmcpset}
%\let\ifpdf\relax
\usepackage[margin=1in]{geometry}
\usepackage{graphicx}
\usepackage{enumerate}
\usepackage{amsthm}
\usepackage{amsmath}

\newcommand{\ds}{\displaystyle}
\newcommand{\half}{\frac{1}{2}}
\newcommand*\Eval[3]{\left.#1\right\rvert_{#2}^{#3}}
\newcommand{\eval}{\biggr\rvert}
%\renewcommand{\vec}[1]{\mathbf{#1}}
\def\EE{{\cal E}}

\setlength{\parindent}{0cm}

\name{}
\class{Physics 111 Section 1}
\assignment{Problem Set 02}
\duedate{September 7, 2016}

\begin{document}

\problemlist{Hamilton's Principle; The Lagrangian and Unconstrained Systems (Reading: Chapter 7.1)}
\textbf{Help:}

\begin{problem}[7.4]
A mass moving along a horizontal plane held at an angle $\alpha$ with respect to the horizontal.\\
Consider a mass $m$ moving in a frictionless plane that slopes at an angle $\alpha$ with the horizontal.  Write down Lagrangian in terms of coordinates $x$, measured horizontally across the slope, and $y$, measured down the slope. (Treat the system as two-dimensional, but include the gravitational potential energy.) Find the two Lagrange equations and show that they are what you should have expected.
\end{problem}


\begin{solution}



\vfill
\end{solution}

\newpage

\begin{problem}[7.8]
Two masses connected by a spring in 1D. In this problem you'll find that the center-of-mass motion of the system is unimportant (there are no external forces), and that the thing we care about is the relative motion of the two particles. This is a great introduction to the central force problems we'll study in chapter 8.\\
\begin{enumerate}
\item Write down the Lagrangian $\mathcal{L}(x_1, x_2, \dot x_1, \dot x_2$) for two particles of equal masses $m_1 = m_2 = m$, confined to the $x$ axis and connected by a spring with potential energy $U = \half k x^2$. [Here $x$ is the extension of the spring, $x = (x_1 - x_2 - l)$, where $l$ is the spring's unstretched length, and I assume that mass 1 remains to the right of mass 2 at all times.] 
\item Rewrite $\mathcal{L}$ in terms of the new variables $X = \half(x_1 + x_2)$ (the CM postion) and $x$ (the extension),and write down the two Lagrange equations for $X$ and $x$.
\item Solve for $X(t)$ and $x(t)$ and describe the motion.

\end{enumerate}

\end{problem}
\begin{solution}



\vfill
\end{solution}
\end{document}
