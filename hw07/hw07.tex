\documentclass[11pt,letterpaper,boxed]{../hmcpsetrhino}
\usepackage[margin=1in]{geometry}
\usepackage{graphicx}
\usepackage{enumerate}
\usepackage{amsthm}
\usepackage{amsmath}

\newcommand{\ds}{\displaystyle}
\newcommand{\half}{\frac{1}{2}}
\newcommand*\Eval[3]{\left.#1\right\rvert_{#2}^{#3}}
\newcommand{\eval}{\biggr\rvert}
\newcommand\Partial[2]{\frac{\partial #1}{\partial #2}}
\let\oldvec\vec
\renewcommand{\vec}[1]{\oldvec{\mathbf{#1}}}
\def\EE{{\cal E}}
\def\Lagr{\mathcal{L}}
\def\Ham{\mathcal{H}}

\name{}
\class{Physics 111 Section 1}
\assignment{Problem Set 07}
\duedate{October 3, 2016}

\begin{document}

\problemlist{Central Forces: Effective Potential}
\textbf{Help:} 

\begin{problem}[i]
Properties of a zero-curl force\\
\begin{problem}[4.43]
In Section 4.8, I claimed that a force $\vec F(\vec r)$ that is central and spherically symmetric is automatically conservative. Here are two ways to prove it:
\begin{enumerate}[(a)]
\item Since $\vec F (\vec r)$ is central and spherically symmetric, it must have the form $\vec F(\vec r) = f(r) \hat r$. Using Cartesian coordinates, show that this implies that $\nabla \times \vec F = 0$.
\item Even quicker, using the expression given inside the back cover for $\nabla \times \vec F$ in spherical polars, show that $\nabla \times \vec F = 0$.
\end{enumerate}
\end{problem}
\vspace{-0.45cm}
\end{problem}
\begin{solution}


\vfill
\end{solution}

\newpage 

\begin{problem}[ii]
Two particles joined by a massless spring\\
\begin{problem}[8.3]
Two particles of masses $m_1$ and $m_2$ are joined by a massless spring of natural length $L$ and force constant $k$. Initially, $m_2$ is resting on a table and I am holding $m_1$ vertically above $m_2$ at a height $L$. At time $t=0$, I project $m_1$ vertically upward with initial velocity $v_0$. Find the positions of the two masses at any subsequent time $t$ (before either mass returns to the table) and describe the motion. [\textit{Hints}: See problem 8.2. Assume that $v_0$ is small enough that the two masses never collide.]
\end{problem}
\vspace{-0.45cm}
\end{problem}
\begin{solution}


\vfill
\end{solution}


\newpage

\begin{problem}[iii]
Effective potential for Hooke's law\\
\begin{problem}[8.13]
Two particles whose reduced mass is $\mu$ interact via a potential energy $U = \half k r^2$, where $r$ is the distance between them.
\begin{enumerate}[(a)]
\item Make a sketch showing $U(r)$, the centrifugal potential energy $U_{cf}(r)$, and the effective potential energy $U_{eff}(r)$. (Treat the angular momentum $l$ as a known, fixed constant.)
\item Find the "equilibrium" separation $r_0$, the distance at which the two particles can circle each other with constant $r$. [\textit{Hint}: This requires that $dU_{eff}/dr$ be zero.] 
\item By making a Taylor expansion of $U_{eft}(r)$ about the equilibrium point $r_0$ and neglecting all terms in $(r-r_0)^3$ and higher, find the frequency of small oscillations about the circular orbit if the particles are disturbed a little from separation $r_0$.
\end{enumerate}
\end{problem}
\vspace{-0.45cm}
\end{problem}

\begin{solution}

\vfill
\end{solution}

\newpage

\begin{problem}[iv]
Effective potential for closed orbits. Hint: you should be able to show that 
\[	U''(r_0) = \frac{(n+2)l^2}{\mu {r_0}^4}\]
\begin{problem}[8.14]
Consider a particle of reduced mass $\mu$ orbiting in a central force with $U = kr^n$ where $kn > 0$.
\begin{enumerate}[(a)]
\item Explain what the condition $kn > 0$ tells us about the force. Sketch the effective potential energy $U_{eff}$ for the cases that $n =2, -1$, and $-3$.
\item Find the radius at which the particle (with given angular momentum $l$) can orbit at a fixed radius. For what values of $n$ is this circular orbit stable? Do your sketches confirm this conclusion?
\item For the stable case, show that the period of small oscillations about the circular orbit is $\tau_{osc} = \tau_{orb}/\sqrt{n+2}$. Argue that if $\sqrt{n+2}$ is a rational number, these orbits are closed. Sketch them for the cases that $n = 2, -1$, and 7.
\end{enumerate}
\end{problem}
\vspace{-0.45cm}
\end{problem}

\begin{solution}

\vfill
\end{solution}
\end{document}
